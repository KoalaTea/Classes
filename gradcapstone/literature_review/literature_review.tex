\section{Introduction}
    Continuous Integration (CI) has grown in popularity, being used in both enterprise and open source projects. CI provides an increase in productivity to programming projects by creating an environment where building, testing, and often deployment is done automatically. When developers do not need to do these tasks manually they can spend more time programming and finding bugs that trigger a failed build or failed test. Most research into CI looks at increasing productivity of developers and the enforcement of development standards. The potential use of CI for security has been mostly overlooked with most research looking at exploiting and protecting a CI system or a CI pipeline and not at the benefits of a pipeline focusing on application security.

	A separate pipeline for security could provide a number of benefits for a security baseline. It could force certain configurations such as Content Security Policy in web applications, or lack of use of depreciated crypto algorithms such as DES or TripleDES\cite{Vehent}. A pipeline that looks only at big win configurations such as these would serve a similar function as Automated Source Code Analysis Tools (ASCAT) do when looking at code meeting project coding
    guidelines\cite{Zampetti}. The pipeline could also include ASCATs that search for security bugs in a warning stage to avoid wasted time with failed builds from false positive. Another potential feature is an inclusion of fuzzers, which were recently used against memory forensic tools to look into anti-forensics techniques through crashing the tools\cite{Case}.

\section{Reviews}
\subsection{Effective use of CI}
	Continuous integration has been a fantastic addition to Software Engineering practices by removing repetitive administrative tasks. The less steps a software engineer has to go through when producing new code, the more time they can spend on adding features, squashing bugs, or otherwise improving the project code base. There is plenty of research done into the use of CI including many case studies, papers and presentations looking at how to setup CI and Continuous Deployment (CD), and papers looking at specific configurations of CI.
	
	CI can be integrated with all sorts of tools to increase productivity. The paper How Open Source Projects use Static Code Analysis Tools in Continuous Integration Pipelines by Fiorella Zampett looks at how Static Code Analysis tools in a CI pipeline effects projects.\cite{Zampetti} The paper found that the most utilized feature of ASCATs was to ensure a project's code was consistent by ensuring the developers coding guidelines were met. The paper also found that ASCATs caused broken builds to be fixed quickly in an average of 8 hours and one build. There are a number of recommendations that this paper provides which outline how to set up ASCATs in a CI pipeline, what to think about when doing so, and what to expect to maintain the ASCAT. Adding a source code analysis tool will help keep a project consistent while helping point out bugs that could be missed by developer made unit tests.

	Two papers look at CI and how it affects projects in general. Continuous Integration and Quality Assurance: A Case Study of Two Open Source Projects by Jesper Holck and Usage, Costs, and Benefits of Continuous Integration in Open-Source Projects by Michael Hilton both look at open source projects and how CI affects them.\cite{Hilton}\cite{Holck} The papers found that CI often replaces the practice of having developers make formal design documents. The developers instead just pick from a list of tasks and works on completing them and merging them into the code base. The papers also found that most developers like CI and plan to use it again in future projects. For projects that did not use CI it was found that the reason was usually just that the developers were not familiar with how to set up and use CI. Both papers found CI to be successful in increasing productivity, causing projects to release twice as often, accept pull requests quicker, and have developers less worried about breaking the project. 

\subsection{Security in CI pipelines}
	Security conferences often look into how to break systems as a way of pointing out the flaws of a configuration. CI servers have also had their share of security professionals testing for exploits and looking at the consequences of compromise. CI has also had a little bit of research into how to secure a CI pipeline.

	A presentation at DEFCON named Exploiting Continuous Integration (CI) and Automated Build Systems talked about the consequences of a CI enabled projects\cite{spaceb0x}. The presentation found that exploiting a repo that holds a CI integration ends with a huge amount of access. If the repo links into an internal CI server, then the attacker ends up with internal network access. If the repo links to a CI server with multiple CI instances or also runs the CD, then the attacker will get more source code and access to the deployment machines because the CI holds a way to connect to the deployment servers. Otherwise the attacker ends up with environment variables which often hold extremely sensitive information.

	A paper by Len Bass called Securing a Deployment Pipeline looks at how to secure a CI pipeline to limit the damage in case of exploitation.\cite{Bass} The paper details a way to break a pipeline down into trusted and untrustworthy parts, segmenting operations until an untrustworthy segment cannot be broken down any more. The CI pipeline then holds parts that are guaranteed to be trustworthy and run as expected, minus specific cases outlined in the paper, and parts that may provide untrustworthy output. By limiting the scope of untrustworthy parts, the rest of the pipeline can run as expected and it is possible to see where the most risk lies. Then the owners of the pipeline can work to limit access to the untrustworthy portions and look into solutions to make those portions trustworthy.

\subsection{Use of CI for Security}
	There is little research into the use of CI for security.[1] One presentation by Mozilla looks at the use of CI to tackle easy fixes in response to their bug bounty program. The presentation looks at using CI to ensure that the production environment contains configurations that mirror best practices for common web application security bugs. Some examples include HSTS is enabled, CSP for XSS bugs, various X-OPTIONS headers, Cookies have secure, Cross origin sharing, and Subresource integrity
    checks. The presentation recommends figuring out a security baseline for a projects CI pipeline, drive testing from the CI pipeline, and empower the team to fix the issues. Another recommendation is not to break on deployment into production as that could break the production site if configured poorly. The end result of mozilla's CI setup was a large drop in bug reports that the CI tests aimed to fix.

\section{Work and method}
	The work into CI so far has shown that CI is useful tool for developers to ease the creation of new features. CI has spread to open source projects and is deployed in most organizations that have at least one large code project. The security side of how a CI is dangerous if exploited and how to secure a pipeline against attacks has had little research. The most interesting thing that I think is lacking is the look into how CI can be used to improve the security the project that it is integrated on.
	
	There are two ways I feel CI could help improve the security of a project. One is in targeting common bugs that are already fixed. Some examples are ensuring that CSP is enabled, HSTS is enabled, parameterized queries are used, binary protections are enabled in compilation scripts, and unsafe functions are not used. Another use could be in including fuzzing in a pipeline. I have not seen any research into automating fuzzing into testing an application. Depending on the project it could be a very useful tool to find bugs.

	I would like to implement a security focused pipeline which will recommend a number of the common bugs to test for. I also want to try to add fuzzing and see how viable it is as a tool in CI. The pipeline will also include some ASCATs, looking at the output from them in a warning state. The output can still be followed up on to decide whether or not it is a false positive.

\bibliography{thebibliography}
